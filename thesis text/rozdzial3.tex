\chapter{Implementacja aplikacji}
\thispagestyle{chapterBeginStyle}
\label{rozdzial3}

\section{Opis technologii}

Do implementacji systemu został użyty język Kotlin w wersji 1.3.61. (aktualizacja wersji i lepszy opis jezyka) Interfejs graficzny zaprojektowano w oparciu o komponenty pochodzące z biblioteki AndroidX oraz Material. Do nawigacji w głównej aktywności aplikacji wykorzystano NavigationUI. Do implementacji bazy danych użyto biblioteki Room(moze coś wiecej) zapewniającej poziom abstrakcji nad SQLite. Wykorzystano bibliotekę Hilt/Dagger w celu wstrzykiwania dependencji w obrębie aplikacji w celu zredukowania tak zwanego ``boilerplate code'' (może więcej).
\newline\newline
W systemie wykorzystano inteligentną opaskę Xiaomi MiBand 3. Komunikacja z nią została zaimplementowana na podstawie nieoficjalnego SDK(większośc trzeba było napisać na nowo lub wyrzucić) opartego o bibliotekę
RxAndroid pozwalającą tworzyć asynchroniczne programy bazujące na wydarzeniach korzystając z obserwowalnych sekwencji.


\section{Omówienie wybranych kodów źródłowych}

