\chapter{Instrukcja obsługi}
\thispagestyle{chapterBeginStyle}
\label{rozdzial4}

W tym rozdziale omówiono wymagania wobec środowiska, w którym będzie instalowana będzie opisana w pracy aplikacja. Przedstawiono procedurę instalacji oraz jak przeprowadzić podstawową konfigurację. W następnej sekcji przedstawiono kilka przykładów użycia aplikacji.

\section{Instalacja i konfiguracja}
\subsection{Wymagania sprzętowe}
Aplikacja została stworzona pod system Android Pie. Do działania jest wymagany moduł Bluetooth w wersji co najmniej 4.0 oraz posiadanie smartbanda MiBand 3.
\subsection{Instalacja}
Aplikację można zainstalować przy użyciu dołączonego do pracy pliku \textit{lockband.apk}. Aby umożliwić instalowanie aplikacji z plików APK należy nadać odpowiedniej aplikacji (w tym przypadku menadżerowi plików) pozwolenie na instalowanie nieznanych aplikacji. Lokalizacja tego ustawienia może się znacząco różnić w zależności od wersji nakładki systemowej. Następnie należy pobrać plik APK na smartfona. W kolejnym kroku należy znaleźć pobrany wcześniej plik korzystając z menedżera plików. Ostatnim krokiem jest stuknięcie w plik APK, by uruchomić instalację.
\subsection{Pierwsze uruchomienie}
- dodać screeny do punktów
\begin{enumerate}
    \item Gdy aplikacja jest uruchamiana po raz pierwszy należy ustawić jej w ustawieniach "Usage stats permission". Jeśli przy uruchomieniu nie ma tego pozwolenia aplikacja przekieruje w odpowiednie miejsce w ustawieniach jednak będzie to wymagało ponownego uruchomienia aplikacji.
    \item Następnie jeśli aplikacja jest uruchamiana z aktywnym "Usage stats permission" to użytkownik jest proszony o pozwolenie na lokalizację. Jest ono niezbędne do poprawnego działania BT.
    \item Kiedy uzyskane zostaną wszystkie potrzebne pozwolenia użytkownik zostaje przeniesiony do aktywności, w której jest tworzone hasło, które będzie wykorzystywane przy odblokowywaniu trybu "Lockdown".
    \item  Użytkownik wprowadza dwa razy hasło i dotyka przycisk (drzwi ze strzałką)
    \item Po utworzeniu hasła użytkownik zostaje przekierowany do aktywności odpowiadającej za parowanie opaski MiBand 3. 
    \item Użytkownik klika przycisk scan for devices
    \item Następnie uruchamiany jest skan urządzeń ble z filtrem na mi band 3
    \item Jeśli urządzenie zostanie odnalezione pojawi się na ekranie. Jeżeli nie zostanie odnalezione należy powtórzyć skan dotykając przycisk scan for devices
    \item Po naciśnięciu na odpowiednie urządzenie na liście znalezionych urządzeń zostaje uruchomiona usługa odpowiadająca za komunikację z MiBand i użytkownik przechodzi do głównego widoku aplikacji
\end{enumerate}

\section{Przykłady użycia}
W tej sekcji przedstawię jak działa aplikacja od strony użytkownika poprzez opis czynności potrzebnych do wykonania określonych zadań, np. sprawdzenie stanu opaski, zmiana blokowanych aplikacji czy sprawdzenie statystyk.
\subsection{Wybór aplikacji do zablokowania}
\subsection{Zmiana hasła użytkownika}
\subsection{Wyświetlenie statystyk aktywności}
\subsection{Wyświetlenie informacji o opasce}
\subsection{Odblokowanie systemu}