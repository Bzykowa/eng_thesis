\chapter{Instrukcja obsługi}
\thispagestyle{chapterBeginStyle}
\label{rozdzial4}

W tym rozdziale omówniono założenia co do środowiska, w którym realizowana będzie instalowana. Przedstawiono procedurę instalacji i wdrożenia systemu. 

\section{Instalacja i konfiguracja}
W tej sekcji omówię wymagania środowiskowe aplikacji oraz kwestie konfiguracyjne jak parowanie opaski czy ustawienie hasła.
\subsection{Wymagania sprzętowe}
System android 9; moduł BT; opaska MiBand 3; 
\subsection{Instalacja}
\subsection{Pierwsze uruchomienie}
- dodać screeny do punktów
\begin{enumerate}
    \item Gdy aplikacja jest uruchamiana po raz pierwszy należy ustawić jej w ustawieniach "Usage stats permission". Jeśli przy uruchomieniu nie ma tego pozwolenia aplikacja przekieruje w odpowiednie miejsce w ustawieniach jednak będzie to wymagało ponownego uruchomienia aplikacji.
    \item Następnie jeśli aplikacja jest uruchamiana z aktywnym "Usage stats permission" to użytkownik jest proszony o pozwolenie na lokalizację. Jest ono niezbędne do poprawnego działania BT.
    \item Kiedy uzyskane zostaną wszystkie potrzebne pozwolenia użytkownik zostaje przeniesiony do aktywności, w której jest tworzone hasło, które będzie wykorzystywane przy odblokowywaniu trybu "Lockdown".
    \item  Użytkownik wprowadza dwa razy hasło i dotyka przycisk (drzwi ze strzałką)
    \item Po utworzeniu hasła użytkownik zostaje przekierowany do aktywności odpowiadającej za parowanie opaski MiBand 3. 
    \item Użytkownik klika przycisk scan for devices
    \item Następnie uruchamiany jest skan urządzeń ble z filtrem na mi band 3
    \item Jeśli urządzenie zostanie odnalezione pojawi się na ekranie. Jeżeli nie zostanie odnalezione należy powtórzyć skan dotykając przycisk scan for devices
    \item Po naciśnięciu na odpowiednie urządzenie na liście znalezionych urządzeń zostaje uruchomiona usługa odpowiadająca za komunikację z MiBand i użytkownik przechodzi do głównego widoku aplikacji
\end{enumerate}
\subsection{Wybór aplikacji do zablokowania}

\section{Przykłady użycia}
W tej sekcji przedstawię jak działa aplikacja od strony użytkownika poprzez opis czynności potrzebnych do wykonania określonych zadań, np. sprawdzenie stanu opaski, zmiana blokowanych aplikacji czy sprawdzenie statystyk.