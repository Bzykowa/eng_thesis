\chapter{Wstęp}
\thispagestyle{chapterBeginStyle}
\label{wstep}

Praca swoim zakresem obejmuje projekt systemu stałej autoryzacji wykorzystujący prostą analizę behawioralną w czasie rzeczywistym na 
podstawie danych gromadzonych przez inteligentną opaskę dla smartfonów działających pod Androidem. Klucze bezpieczeństwa są rzadkim zjawiskiem w 
mobilnych systemach. Niska popularność tego rozwiązania może wynikać z ceny tych urządzeń, często niekompatybilnych ze smartfonem metodach połączenia oraz 
zastępowalności innymi metodami wieloskładnikowej autoryzacji.
\newline\newline
\indent Celem pracy jest zaprojektowanie i stworzenie aplikacji o następujących założeniach funkcjonalnych: 
\begin{itemize}
	\item Aplikacja wykorzystuje inteligentną opaskę (pot. smartband) jako klucz bezpieczeństwa;
	\item Aplikacja pozwala blokować dostęp do innych aplikacji;
    \item Aplikacja komunikuje się z inteligentną opaską przy wykorzystaniu protokołu Bluetooth Low Energy;
    \item Aplikacja analizuje zachowanie użytkownika w czasie rzeczywistym;
    \item Aplikacja działa w tle;
    \item Aplikacja pozwala wybrać, które aplikacje obejmie ochroną;
    \item Wszystkie dane użytkownika są przechowywane lokalnie;
    \item Aplikacja jest odporna na popularne ataki.
\end{itemize}

\indent Praca składa się z sześciu rozdziałów. W rozdziale \ref{rozdzial1} przedstawiono dogłębnie problem autoryzacji przy użyciu sprzętowych kluczy w smartfonach oraz braku prywatności wrażliwych danych gromadzonych przez smartbandy. 
Omówiono szczegółowo dane pobierane z sensorów w opasce oraz smartfonie. Scharakteryzowano zdarzenia, przy których ograniczany jest dostęp do 
urządzenia. Opisano także rozwiązania podjęte w celu uniemożliwienia sabotażu aplikacji. Przeprowadzono również analizę porównawczą istniejących rozwiązań z realizowanym 
systemem.

W rozdziale \ref{rozdzial2} przedstawiono projekt aplikacji w notacji UML. Wykorzystano diagramy przypadków użycia, komponentów oraz stanów. Przedstawiono dokładne scenariusze do przypadków użycia. Omówiono też projekt bazy danych.

W rozdziale \ref{rozdzial3} określono technologie użyte w implementacji aplikacji: wybrany język programowania oraz wykorzystywane biblioteki. Opisano w pseudokodzie i omówiono algorytmy blokujące dostęp do aplikacji. Wyczerpująco opisano protokoły komunikacji z opaską. Określono także w jaki sposób są analizowane dane o aktywności.

W rozdziale \ref{rozdzial4} przedstawiono wymagania aplikacji wobec środowiska. Określono także sposób instalacji oraz konfiguracji aplikacji. Rozdział zawiera również przykłady działania dla użytkownika.

Końcowy rozdział jest podsumowaniem uzyskanych wyników.



