\chapter{Wstęp}
\thispagestyle{chapterBeginStyle}

Praca swoim zakresem obejmuje projekt systemu stałej autoryzacji wykorzystujący prostą analizę behawioralną w czasie rzeczywistym na 
podstawie danych gromadzonych przez inteligentną opaskę dla smartfonów działających pod Androidem. Klucze bezpieczeństwa są rzadkim zjawiskiem w 
mobilnych systemach. Niska popularność tego rozwiązania może wynikać z ceny tych urządzeń, często niekompatybilnych ze smartfonem metodach połączenia oraz 
zastępowalności innymi metodami wieloskładnikowej autoryzacji.
\newline\newline
\indent Celem pracy jest zaprojektowanie i stworzenie aplikacji o następujących założeniach funkcjonalnych: 
\begin{itemize}
	\item użycie inteligentnej opaski (pot. smartband) jako klucza bezpieczeństwa,
	\item zapewnienie prywatności gromadzonych danych,
	\item monitorowanie zachowań użytkownika w czasie rzeczywistym,
	\item zabezpieczenie urządzenia mobilnego przed uzyskaniem dostępu przez osoby niepowołane,
	\item stworzenie praktycznego systemu stałej autoryzacji dedykowanego dla urządzeń mobilnych.
\end{itemize}

\indent Istnieje garstka systemów o podobnej funkcjonalności: Gadgetbridge, aplikacja open-source zastępująca korzystające z chmury oprogramowanie producentów wybranych opasek, przy czym nie zapewnia 
żadnych funkcji z zakresu bezpieczeństwa; Haven, aplikacja monitorująca czynniki środowiskowe w celu 
uzyskania dowodów ataku, jednakże nie zabezpieczająca przed jego skutkami oraz Android Management API, usługa dla przedsiębiorstw umożliwiająca zarządzanie grupą urządzeń oraz zabezpieczenie ich poprzez zdalne wydawanie "polityk", 
przy czym rozwiązanie to nie jest aplikowalne poza środowiskami biznesowymi.
\newline\newline
\indent Praca składa się z czterech rozdziałów. W rozdziale pierwszym przedstawiono dogłębnie problem autoryzacji przy użyciu sprzętowych kluczy w smartfonach oraz braku prywatności wrażliwych danych gromadzonych przez smartbandy. 
Omówiono szczegółowo dane pobierane z sensorów w opasce oraz smartfonie. Scharakteryzowano zdarzenia, przy których ograniczany jest dostęp do 
urządzenia. Opisano rozwiązania podjęte w celu uniemożliwienia sabotażu aplikacji. Przeprowadzono analizę porównawczą istniejących rozwiązań z realizowanym 
systemem.
\newline\newline
\indent W rozdziale drugim przedstawiono szczegółowy projekt aplikacji w notacji UML. Wykorzystano diagramy przypadków użycia, klas, aktywności, sekwencji oraz stanów. 
Omówiono dokładnie projekt bazy danych. Wyczerpująco opisano protokoły komunikacji z opaską. Opisano w pseudokodzie i omówiono algorytmy blokujące dostęp do aplikacji. 
\newline\newline
\indent W rozdziale trzecim określono technologie użyte w implementacji aplikacji: wybrany język 
programowania, wykorzystywane biblioteki, model opaski oraz typ bazy danych. Przedstawiono dokumentację 
techniczną wybranych kodów źródłowych. 
\newline\newline
\indent W rozdziale czwartym przedstawiono wymagania aplikacji co do środowiska. Określono także sposób instalacji oraz konfiguracji aplikacji. Rozdział zawiera również przykłady działania dla 
użytkownika.
\newline\newline
\indent Końcowy rozdział jest podsumowaniem uzyskanych wyników.



